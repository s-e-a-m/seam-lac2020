% Template LaTeX file for LAC-20 papers
%
% To generate the correct references using BibTeX, run
%     latex, bibtex, latex, latex
% modified...
% - from DAFx-00 to DAFx-02 by Florian Keiler, 2002-07-08
% - from DAFx-02 to DAFx-03 by Gianpaolo Evangelista
% - from DAFx-05 to DAFx-06 by Vincent Verfaille, 2006-02-05
% - from DAFx-06 to DAFx-07 by Vincent Verfaille, 2007-01-05
%                          and Sylvain Marchand, 2007-01-31
% - from DAFx-07 to DAFx-08 by Henri Penttinen, 2007-12-12
%                          and Jyri Pakarinen 2008-01-28
% - from DAFx-08 to DAFx-09 by Giorgio Prandi, Fabio Antonacci 2008-10-03
% - from DAFx-09 to DAFx-10 by Hannes Pomberger 2010-02-01
% - from DAFx-10 to DAFx-12 by Jez Wells 2011
% - from DAFx-12 to DAFx-14 by Sascha Disch 2013
% - from DAFx-15 to DAFx-16 by Pavel Rajmic 2015
% - from DAFx-16 to IFC-18 by Romain Michon 2018
% - from IFC-18 to LAC-19 by Romain Michon 2019
% - from LAC-19 to LAC-20 by Jean-Michaël Celerier 2020
%
% Template with hyper-references (links) active after conversion to pdf
% (with the distiller) or if compiled with pdflatex.
%
% 20060205: added package 'hypcap' to correct hyperlinks to figures and tables
% use of \papertitle and \paperauthorA, etc for same title in PDF and Metadata
%
% 1) Please compile using lualatex, latex or pdflatex.
% 2) If using pdflatex, you need your figures in a file format other than eps!
% e.g. png or jpg is working
% 3) Please use "papertitle" and "pdfauthor" definitions below

%-------------------------------------------------------------------------------
%  !  !  !  !  !  !  !  !  !  !  !  ! user defined variables  !  !  !  !  !  !
% Please use these commands to define title and author(s) of the paper:
\def\papertitle{SEAM PROJECT - Sustained Electroacoustic Music}
\def\paperauthorA{Giuseppe Silvi}
\def\paperauthorB{Davide Tedesco}
%\def\paperauthorC{Author Three}
%\def\paperauthorD{Author Four}

% Authors' affiliations have to be set below

%-------------------------------------------------------------------------------
\documentclass[twoside,a4paper]{article}
\usepackage{LAC-20}
\usepackage{amsmath,amssymb,amsfonts,amsthm}
\usepackage{euscript}
\usepackage{ifpdf}
\usepackage{ifluatex}
\usepackage{ifxetex}

\usepackage{color}
\usepackage{listings}
\definecolor{mygrey}{rgb}{0.96,0.96,0.96}
\lstset{
  tabsize=4,
  basicstyle=\ttfamily,
  backgroundcolor=\color{mygrey},
  captionpos=b,
  breaklines=true
}

\usepackage[italian,english]{babel}
\usepackage{caption}
\usepackage{subfig, color}
\setcounter{page}{1}
\ninept

\usepackage{times}
% pdf-tex settings: detect automatically if run by latex or pdflatex
\ifluatex
  \usepackage[
    pdftitle={\papertitle},
    pdfauthor={\paperauthorA, \paperauthorB},%, \paperauthorC, \paperauthorD},
    colorlinks=false, % links are activated as colror boxes instead of color text
    bookmarksnumbered, % use section numbers with bookmarks
    pdfstartview=XYZ % start with zoom=100% instead of full screen; especially
  ]{hyperref}        % useful if working with a big screen :-)

  \edef\pdfcompresslevel{\pdfvariable compresslevel}
  \pdfcompresslevel=9
  \usepackage{graphicx}

  \usepackage[figure,table]{hypcap}
  \usepackage{fontspec}
\else
  \ifxetex
    \usepackage[
      pdftitle={\papertitle},
      pdfauthor={\paperauthorA, \paperauthorB},%, \paperauthorC, \paperauthorD},
      colorlinks=false, % links are activated as colror boxes instead of color text
      bookmarksnumbered, % use section numbers with bookmarks
      pdfstartview=XYZ % start with zoom=100% instead of full screen;
    ]{hyperref}        % especially useful if working with a big screen :-)

    \pdfcompresslevel=9
    \usepackage{graphicx}

    \usepackage[figure,table]{hypcap}
    \usepackage{fontspec}
  \else
    \usepackage[utf8]{inputenc}
    \usepackage[T1]{fontenc}
    \ifpdf % compiling with pdflatex
      \usepackage[pdftex,
        pdftitle={\papertitle},
        pdfauthor={\paperauthorA, \paperauthorB},%, \paperauthorC, \paperauthorD},
        colorlinks=false, % links are activated as colror boxes instead of color text
        bookmarksnumbered, % use section numbers with bookmarks
        pdfstartview=XYZ % start with zoom=100% instead of full screen;
      ]{hyperref}        % especially useful if working with a big screen :-)
      \pdfcompresslevel=9
      \usepackage[pdftex]{graphicx}
      \usepackage[figure,table]{hypcap}
      \DeclareGraphicsExtensions{.png,.jpg,.pdf}
    \else % compiling with latex
      \usepackage[dvips]{epsfig,graphicx}
      \usepackage[dvips,
        colorlinks=false, % no color links
        bookmarksnumbered, % use section numbers with bookmarks
        pdfstartview=XYZ % start with zoom=100% instead of full screen
      ]{hyperref}
      % hyperrefs are active in the pdf file after conversion
      \usepackage[figure,table]{hypcap}
      \DeclareGraphicsExtensions{.eps}
    \fi
  \fi
\fi

\title{\papertitle}

%-SINGLE-AUTHOR HEADER STARTS (uncomment below if your paper has a single author)
% \affiliation{
% \paperauthorA \,\sthanks{This work was supported by the XYZ Foundation}}
% {\href{https://scrime.u-bordeaux.fr}{SCRIME} \\ Université de Bordeaux, France \\
% {\tt \href{mailto:ping@linuxaudio.org}{ping@linuxaudio.org}}
% }
%-----------------------------------SINGLE-AUTHOR HEADER ENDS-------------------

%-TWO-AUTHOR HEADER STARTS (uncomment below if your paper has two authors)------
 \twoaffiliations{
 \paperauthorA \,\sthanks{Adjunct Professor in Interpretation and Performance of
                          Electroacustic Music at the Electronic Music Department
                          (SMERM) of the Conservatory of Music “Santa Cecilia”,
                          Rome, Italy}}
 {\href{https://www.conservatoriosantacecilia.it/}{SMERM} \\
 Conservatory of Music “Santa Cecilia”, Rome, Italy \\
 {\tt \href{mailto:grammaton@me.com}{grammaton@me.com}}
 }
 {\paperauthorB \,\sthanks{Graduate Student of the Electronic Music Department
                           (SMERM) of the Conservatory of Music “Santa Cecilia”,
                           Rome, Italy }}
 {\href{https://www.conservatoriosantacecilia.it/}{SMERM} \\
 Conservatory of Music “Santa Cecilia”, Rome, Italy \\
 {\tt \href{mailto:davide.tedesco.rome@gmail.com}{davide.tedesco.rome@gmail.com}}
 }
%-------------------------------------TWO-AUTHOR HEADER ENDS--------------------

%-THREE-AUTHOR HEADER STARTS (uncomment below if your paper has three authors)--
% \threeaffiliations{
% \paperauthorA \,\sthanks{This work was supported by the XYZ Foundation}}
% {\href{https://scrime.u-bordeaux.fr}{SCRIME} \\ Université de Bordeaux, France \\
% {\tt \href{mailto:ping@linuxaudio.org}{ping@linuxaudio.org}}
% }
% {\paperauthorB \,\sthanks{This guy is a very good fellow}}
% {\href{https://ccrma.stanford.edu}{CCRMA} \\ Stanford University, USA \\
% {\tt \href{mailto:lac@ccrma.stanford.edu}{lac@ccrma.stanford.edu}}
% }
% {\paperauthorC \,\sthanks{Illustrious contributor}}
% {\href{http://www.musikwissenschaft.uni-mainz.de/Musikinformatik/}{Johannes
%                             Gutenberg University (JGU)} \\  Mainz, Germany\\
% {\tt \href{mailto:lac@uni-mainz.de}{lac@uni-mainz.de}}
% }
%-------------------------------------THREE-AUTHOR HEADER ENDS------------------

%-FOUR-AUTHOR HEADER STARTS (uncomment below if your paper has four authors)----
%\fouraffiliations{
%\paperauthorA \,\sthanks{This work was supported by the XYZ Foundation}}
%{\href{https://scrime.u-bordeaux.fr}{SCRIME} \\ Université de Bordeaux, France \\
%{\tt \href{mailto:ping@linuxaudio.org}{ping@linuxaudio.org}}
%}
%{\paperauthorB \,\sthanks{This guy is a very good fellow}}
%{\href{https://ccrma.stanford.edu}{CCRMA} \\ Stanford University, USA \\
%{\tt \href{mailto:lac@ccrma.stanford.edu}{lac@ccrma.stanford.edu}}
%}
%{\paperauthorC \,\sthanks{Illustrious contributor}}
%{\href{http://www.musikwissenschaft.uni-mainz.de/Musikinformatik/}{Johannes
%                            Gutenberg University (JGU)} \\  Mainz, Germany\\
%{\tt \href{mailto:lac@uni-mainz.de}{lac@uni-mainz.de}}
%}
%{\paperauthorD \,\sthanks{Thanks to the predecessors for the templates}}
%{\href{https://c-base.org/}{C-Base} \\ Berlin, Germany \\
%{\tt \href{mailto:lac@c-base.com}{lac@c-base.com}}
%}
%-------------------------------------FOUR-AUTHOR HEADER ENDS-------------------

\begin{document}

\maketitle

%-------------------------------------------------------------------------------
%-------------------------------------------------------------------------------
%-------------------------------------------------------------------------------

\begin{abstract}

The musical composition is close to a \emph{point break}: almost one hundred
years ago Ottorino Respighi introduced a recorded media into his orchestral
composition \emph{I Pini di Roma}  \cite{ropr25} and, even today, we don't have
a shared consolidate electroacoustic practice to play it likewise the orchestral
one. Someone does it better than others, by its equilibrium between knowledge
and consciousness. After all, it is only a recorded bird sound to be placed
inside an orchestra, not a virtuoso part to be played on a handmade custom
electroacoustic instrument disappeared from the earth except by memories and
score notes. The problem is more serious and profound if we consider that most
of today's electroacoustic manipulators don't know who Respighi was, what
happened after him and what are the differences between his pioneer usage of
recordings, instead of the later compositional purpose usage made by John Cage
\cite{cjil39}. Something must change to introduce a way that conducts a
consolidation practice on electroacoustic literature.

\end{abstract}

%-------------------------------------------------------------------------------
%-------------------------------------------------------------------------------
%-------------------------------------------------------------------------------

\section{Introduction}
\label{sec:intro}

\emph{Sustained Electro-Acoustic Music} is a project inspired by Alvise Vidolin
and Nicola Bernardini's article \cite{bevi05} on \emph{live electroacoustic music
sustainability}. In their text, they point at multiple technical faces of the
sustainability problem such as technological, notational or general conception
issues. Even if the article aforementioned focuses only on \emph{live}
electroacoustic music, the concept of sustainability applies to any kind of
documented music that uses electroacoustic environments including therefore the
acousmatic works, instruments mixed with tape and structured amplified works.
This will be the purpose of the presented text.

The ambition of this project is to grow the interpretation and the electroacoustic
musical practice with the consciousness of the electronic and informatics problems
that had made arduous to approach this music and prevented the growth of
interpretative thinking. It is possible, with a community structure, to determine,
build and stratify interpretation of musical core, the repertoire, concealing the
environment-related technological issues. They are instruments, not the music
itself, after all.

When we refer to a virtuoso musician, we often point at a violinist or a piano
player: someone who intensely practice on his instrument. This is the central
point: Does the violinist builds its violin every time he approaches a new
composition? Does the pianist? The electroacoustic musician does it, every time.

%The Problem section will introduce the definition of general issues and actual
%circumstances. After the description of the SEAM project, there are three
%sections in which a starting idea of sustainability is applied and described in
%three different processes.

%-------------------------------------------------------------------------------
%-------------------------------------------------------------------------------
%-------------------------------------------------------------------------------

\section{Problems}
\label{sec:problems}

The electroacoustic music culture was born in a daily changing context. The
sustainability of what the electroacoustic musicians and composers were doing,
during the years, wasn't an interesting and useful point during the realisation
of the compositions.

\begin{quote}
Interpretation is a way to overcome the technological obsolescence that every
computer musician knows very well. [\ldots] In the beginning of IRCAM, no one was
aware of the seriousness of the problem: the works produced in the 1980s were
made with a total lack of concern for this issue or with an optimistic technophily.
We realized the problem later, in the beginning of the 21st century \cite{lem16}.
\end{quote}

A point consists exactly in the necessity of a definition of computer music or
more generally computer-something, today. \emph{Computer Music} was to mean
something when differences between technologies started to be also compositional
and musical differences. Today's evidence is that there's no music without the
computer and so far we need to move on, to change, develop and solve other
problems with a proper language, otherwise, the situation will remain similar to
decades ago. Sustainability is an intricate and complex concept, and music
sustainability sounds like an abstract problem, applied to an abstract thing
only for a small number of people, such as an abstract community not related to
the mass. Again, we acknowledge that mass-media, mass-culture, mass-society-things,
are no place for the \emph{sustained people}.

Traditionally, music composition developed through an interdisciplinary approach
to research on sound and perception and writing itself. In other words, writing
something that pushes the writing itself into becoming writing, towards the best
comprehension of something. If actual music is afflicted or not by the contemporary
and electroacoustic music issues, it is an ordinary question, but the evidence
that musical thinking changed through the electroacoustic thought is an undeniable
fact. Music was changed inexorably after the introduction of electronics and
informatics in composition, as well as the way how it has transformed the approach
to playing and production. We are not speaking about the inevitable technologic
half of those facts, but of the musical one, built on literature and interpretation.

\begin{quote}
To create its repertoire, the institute asked composers to write works interacting
with the institute's research departments \cite{lem16}.
\end{quote}

The mutation was deep enough to change some general directions, which turned out
to be oriented to technical issues with technical approaches and solutions. That
too technical, but less musical, approach has sliced the musical practice and
composition (even the technological one) from the practice of an instrument (even
the technological one also in this case).

%-------------------------------------------------------------------------------
%-------------------------------------------------------------------------------
%-------------------------------------------------------------------------------

\section{Neatly Layering}
\label{sec:layering}

Deutsche Grammophon released three interpretations of Beethoven's Complete
Symphonies of the four made by the conductor Herbert von Karajan in less than
35 years \cite{rrrnyt}. Each of those boxsets is a collection of reproduction,
not the music itself. We consider it a huge resource of thinking, (Beethoven's
thinking through the Karajan's one) not a huge resource of music itself. Every
man who has listened to Beethoven's music in a concert hall knows perfectly that
his music can't fit in a box that can be placed in a hand. This point of view is
not in coincidence with the discographic purpose that it was built for, but it
doesn't matter. The point is that we have stratified musical thinking and
listening attitude on Beethoven's music through interpretations of his music.
We do have not rewritten his music each time and we have not built his instruments
each time from ground zero. Is it a technological fact? A musical one? Both of
them.

Luigi Nono's repertoire is not on a triple boxset of no one. It is on paper in
the best-case scenario \cite{raprmt}. The \emph{Archivio Luigi Nono} does an
immense musicological and production work by keeping and preserving Nono's works.

His lately composed music, like \emph{Risonanze Erranti} (mentioned later in this
article), in which half of the ensemble not to consist in traditional acoustical
instruments, but in \emph{Live Electronics Instruments}, dated the '80s and not
fully described and neither sustained through the years, leads us to another
question: what can we study and interpret those instruments?

% What can we study and interpret of his lately composed music, like
% \emph{Risonanze Erranti} (mentioned later in this article), in which half of the
% ensemble not to consist in traditional acoustical instruments, but in
% \emph{Live Electronics Instruments}, dated the '80s and not fully described and
% neither sustained through the years? Some people have memories of those
% disappeared instruments from musical daily doing and, after all, they have
% directly worked with Nono and can accurately describe and share what happened
% and what can today happen.

\begin{quote}
In the classical music context, a musical interpretation requires the ability to
read the music (knowing the vocabulary) and to understand the text (knowing the
syntax). It also means mastering its instrument (it takes years of practice to
make a virtuoso), interpreting the composer's will (knowing the stylistic context).
Finally, the musician should be able to perform the music in concert, interacting
with the audience, the hall, and the other performers \cite{lem16}.
\end{quote}

Looking at the Post-Graduate Doctoral offers for an electroacoustic musician
career all over the world, there are many \emph{interactive-all-you-can-think-about}
positions, but nothing about practising the electroacoustic repertoire. There are
a lot of \emph{Machines (that are) Learning} something, somewhere. All over the
world, the music industry conceived the purpose of doing music, with or without
musical problems to solve. During that well-studied interaction learning the art
of entertainment, where the industry is god, and \emph{God is a DJ}, meanwhile,
it grows also a repertoire of music that we must consider the core of the actual
musical thinking and conception, that will disappear in a few years if not
\emph{sustained}. Not the written papers, neither the recordings of that
repertoire. We have archives with \emph{clouds} for that, and some \emph{machines}
to manage and take care of that, maybe. But it will disappear the practice, the
interpretation, the sensibility and the musical thinking itself from roots of
musical comprehension, which are at the same time the roof of composing, the
inspiring starry sky. And there will be no place to store these human-related
aspects. If there are clouds, they are grey and full of rain.

Here are the focal points. What will become the electroacoustic music repertoire
if not the one played in the concert hall? Why we do concentrate too many
resources and time on technical problems and not on musical interpretations and
playing practices of repertoire?

%-------------------------------------------------------------------------------
%-------------------------------------------------------------------------------
%-------------------------------------------------------------------------------

\section{The Seam Community}
\label{sec:seam}

From seam meaning:

\begin{quote}
\begin{it}
A line where two pieces of fabric are sewn together\ldots \\
An underground layer of a mineral such as coal or gold: the buried forests
became seams of coal\ldots\\
Join with a seam.
\end{it}
\end{quote}

We have to study Vidolin's gestures to understand Nono, to have a clear sight on
our music through an era and join literature and practice with a seam. Vidolin is
for Nono what Karajan was for Beethoven: time, consciousness and thinking. We
need his work to know what was happening, what we have to do, what is necessary
and what doesn't matter. And that is we have to do, seam it just one time,
forever. Refine it, maintain it, and again realise it, through practice, forever.
Neatly layering people's knowledge and thinking is the only way to hold back and
preserve what we are loosing, preventing music from being a boxset of objects
without the consciousness of music that they represent.

To prevent catastrophic regression of musical thinking we must consider that
there are few dogmatic concepts to build, re-build and sustain an
\emph{electroacoustic repertoire}:

\begin{enumerate}
  \item Open and Be Open
  \item Don't Repeat Yourself
  \item Think and Act as Community
\end{enumerate}

\textbf{SEAM is an Open, DRY, Community.} People inside SEAM will share their
knowledge to weld words, papers and literature with meaning.

These are the SEAM organisation coordinates:
\begin{itemize}
\item \url{http://s-e-a-m.github.io}
\item \url{http://seam-world.slack.com}
\end{itemize}

There are notably predecessors of this kind of initiative, with more personal
oriented use, some of them has inspired this project, like the Miller Puckette's
repository\footnote{\url{http://msp.ucsd.edu/pdrp/latest/files/}}. We hope the
public domain community profile of SEAM can include some of those precious
wizards contributes, in a more community sense, to avoid the misunderstanding
of literature. An only-tech reading can bring to wrong interpretations even for
great tech minds. That's how Puckette \cite{mp01} resolve a crucial description
of the \emph{Dialogue}:

\begin{quote}
This piece in its published form is performed by one clarinetist accompanied by
a tape of the same clarinetist.
\end{quote}

It is not accompanied, it is a dialogue.

%-------------------------------------------------------------------------------

\subsection{SEAM Instruments}

Developing the concepts of the instrument and instrumentalist to the combined
form of those into interpretation, \cite{lem16,mp01,savi85} requires the
overcoming of obsolete parallelism: the computer music performer as an artisan
of \emph{new-luthiery}. There is not a sustainability conception under the
deception of that wrong and obsolete metaphor. Each \emph{luthiery} is new, it
evolves with musical needs. Each instrument has his inventor and his virtuoso,
but in musical history, those people never coincided. The best instruments were
conceived from men entirely devoted to the conception of something unique. The
best virtuoso took those instruments to unveil their prospective.

%\begin{quote}
%For computer music, things are slightly different because of the nature of the
%“instrument”. There is an extra step: constructing the instrument. In this
%sense the computer music performer is also his own instrument-builder (luthier).
%Moreover, there is no school or conservatory to learn how to become a computer
%virtuoso today.
%\end{quote}

During the lessons in Rome Conservatory in which \emph{SEAM} was born and its
related problems were shared with classes to sensitize students to community
work, the core software used to explode issues was
\emph{Faust}\footnote{\url{https://faust.grame.fr}}. This wasn't a restriction,
it was a preference. Text-based DSP offers the deepest learning experience and
great expressivity and readability. \emph{Faust} could be written to educate a
musician at the same time with computation versatility and efficiency. The
\emph{Faust libraries} concept is useful to focus on write once and read forever
code. We think \emph{Faust} itself represents a rather concept of electroacoustic
sustainability. Thinking, for example, at the \emph{filters.lib} and at the names
that contributed to the enrichment of speculation around each object, make us
wish to a musical interest capable to do community more than with the adoption
of other software.

Instruments carved by musical ideas on readable text (code) becomes a
sub-literature in which each brick maintain the power of the source code, the
clarity of an equation, the efficiency of the continuous development, the
reusability of a word in different contexts.

%--------------------------------------------
%----------------larghezza massima del codice
\begin{lstlisting}
import("stdfaust.lib");
import("../faust-libraries/seam.lib");
\end{lstlisting}

The \emph{SEAM library} local importing points to other libraries catalogued by
arguments, like in \emph{Standard Faust Libraries}.

Actually there are five different libraries:
\begin{description}
  \item[seam.lib] contains general functions and the pointers to each specific
  library. It may also comprehend the custom performative environment definition,
  as it could be for the inputs and the outputs, the setup parameters and the
  performative controls.
  \item[gerzon.lib] contains early Michael Gerzon works, his core concepts of
  spatialization and stereophony, that conducted him to conceive the Ambisonic
  technology. In a sustained environment, the role of this library is to avoid
  misunderstanding of what stereo is \emph{stereo} is \cite{ab58} and what we
  are loosing in the electroacoustic staging perception.
  \item[hardware.lib] contains hardware-related functions like MIDI mapping and
  I/O assignment to an audio interface, with a routing strategy to connect
  instruments to real-world hardware with a graphical user interface to map
  routing.
  \item[measurement.lib] contains some audio analysis strategy to define musical
  display feature for audio inspection, such as integrated measurement and
  loudness monitoring, that are indispensable tools for today staging of the
  public addressed music.
  \item[nono.lib] is the first author-related library that points to contain
  \emph{Live Electronics Instruments}. The idea is to collect instruments into
  the library and use them, work by work, in hardware-like approach. The
  \emph{nono.lib} should contain reusable instruments typical of his literature
  like the Harmonizer, the Halaphon, and so on, directly called back into the
  performance environment of each work, to enforce the reusability and the
  sustainability of those instruments.
\end{description}

Faust is a great tool and we are proud users of it, nevertheless, a studied
choose of the proper tool is required for each specific case. Sustaining of
proper choosing is most important than the comfort of the preferred tools. As
proposed to \emph{max}-addicted students during lessons, a \emph{library}
approach, like the \emph{Faust} one, must be ever incentivized.

%-------------------------------------------------------------------------------

\subsection{SEAM Topology}

Referring to the electroacoustic music literature, where the substantial
difference with the acoustical one is an inevitable continuously changing of the
environment, we prefer to use the topology classification in place of typology
one. A typology classification is, according to general type, used where
characteristics of something are fixed and produce a catalogue of things. A
topology classification considers instead the time-space characteristics of
shape and permits the time variance of the environments. We classify three
topologies of the electroacoustic music in literature:

\begin{description}
  \item[The undocumented] where composers use only word description to generate
  environment and circumstances;
  \item[The \emph{hole-word}] where the score has deep technical documentation
  but listing names of undocumented instruments. Without musicological
  methodologies, frequently with names without a specific meaning;
  \item[The porting] where informatics translations between languages or
  informatics technologies are based on literature and shared knowledge.
\end{description}

The identification of topological classes in place of typological forms is
necessary to subordinate technology-matter to the musical practice and poetics.
%All three topologies are reduced to technological circumstances, not related to
%the musical content and form, but with the sustainability strategies routes to
%the music.

%-------------------------------------------------------------------------------
%-------------------------------------------------------------------------------
%-------------------------------------------------------------------------------

\section{Write the undocumented}
\label{sec:writing}

\emph{The undocumented} are the first topology class we approach. It holds all
works in which composers used only word descriptions to portray the electroacoustic
performing environment, with the rules and circumstances needed. Like Ottorino
Respighi does with \emph{I Pini di Roma} \cite{ropr25} at the very beginnings,
many composers until now never documented their works with specific usages of the
technologies at their disposal. There are tons of scores that implicitly involves
particular amplification, or complex electroacoustic staging, only by words
description.

%-------------------------------------------------------------------------------

\subsection{1969, \emph{I am Sitting in a Room}, Alvin Lucier}

Speaking at beginner music students about Lucier's \emph{I am sitting in a room},
is a kind of sharing of a multilevel experience. There are a lot of access layers
each with different bits of knowledge requirements. One of these, of course, is
how you can do it today.

The score state a text to be read, it explains what is going to happen and why,
so the process unveils the process itself. The acoustical properties of the
space transform the speech. The “resonant frequency of the room reinforces
themselves”, while the others are absorbed, they are attenuated by space. Space
as an instrument to be played and articulated by time.

\begin{quote}
At the time of composition, the only way to realize the score of \emph{I am
sitting in a room} was with tape: using two recorders the text was recycled and
re-recorded, and then all the version were spliced together chronologically.
Concert performances consisted of playing back this composite tape [\ldots] In
the heyday of "live Electronic Music" [\ldots] the piece \emph{could} have been
performed live [\ldots] but to do so would have been to miss a subtle but
important detail: "I am sitting in a room \emph{different} from the one you are
in now." [\ldots] \emph{I am sitting in a room} conveys this sense of rightness
in a way that transcends the mechanism, phenomena, and text of the piece. It
pulls the listener along with process that, whether understandable or not, seems
perfectly natural, totally fascinating, intensely personal, and poignantly
musical. \cite{alCD90}
\end{quote}

Today the work could be realised with live-electronics, without interruption
between cycles. It requires a simple delay line, sized as much as the statement,
to be infinitely recycled. Again, the sensibility that had characterised the
Lucier's Era, today, is overwhelmed by anxiety and incapacity to observe something
in time. The time-lapse perception model is a \emph{state of mind} constriction,
so, with the maximum technological support of an infinite digital delay line,
without the necessity of a full perception, \emph{I am sitting} could be a
surgical time-waste, at best quality, of course. The idea of space as an
instrument expressed by this apostolic work requires ears and fingers twisted
in a full participated perception of time-space mutation during the performance.

%\begin{figure}[ht]
%\centerline{\includegraphics[scale=0.5]{img/BFMT-GUI}}
%\caption{\label{lais-GUI}{\it I am sitting… B-Format GUI}}
%\end{figure}

The very deep sustainability problem of that work isn't technical. It is a
simple process. The very deepest problem is sensibility. The worst thing that
can happen to the \emph{process-music} is the perfect process execution without
the music. To seam process and music we need to unfold ears and minds to the
Lucier's perception and sensibility. How? Doing it, like he exactly suggested
fifty years ago: practising.

\begin{quote}
Make versions in which one recorded statement is recycled through many rooms.
Make versions using one or more speakers of different languages in different
rooms. Make versions in which, for each generation, the microphone is moved to
different parts of the room or rooms. Make versions that can be performed in
real-time. \cite{lais69}
\end{quote}

The many versions proposed by the author in the music score point to multiple
cases of electroacoustic staging. It is a \emph{free-your-electroacoustic-fantasy}
statement, typical of the end of the sixties, unfortunately forgot the day after.
To reset future people's perception by now, young musicians should do that for
years, never mapping one single patch in \emph{max4live}. They need an instrument
to practise music.

The offline process remains, like the original statement says, really unchanged.
A double recording apparatus, of any nature, and a chair to sit down and practise.

%--------------------------------------------
%----------------larghezza massima del codice
\begin{lstlisting}
main = vgroup("[01] Check both boxes to
       start", *(L) : de.delay(maxdel, D-1))
  with{
    maxdel = ma.SR *(180);
    I = int(checkbox("[01] Uncheck me after
        the incipit"));
    C = (I-I') <= 0; // Clear del
    D = (+(I):*(C))~_; // Compute del time
    L = int(checkbox("[02] I am Sitting...
        Uncheck me at the end"));
  };
\end{lstlisting}

The strategy adopted to create a real-time performative environment for \emph{I
am sitting} is to count the samples of the duration of the initial statement and
pass that count to the delay time. The code proposed here is only to evidence
the straightforward writing in \emph{Faust}, underlining that part of the code
is stolen from the \emph{Faust} manual itself.

We propose three different ready-to-fight real-time environments, to practice
with the musical behaviour of the piece. The first is one-in-one-out, easy to
setup. The second is a stereophonic version, where, per stereo sound, we refer
to an unbreakable experience of acoustical listening. \cite{ab58} The third is
four-channel ambisonic version, usable by who four-dimensionally thinks
space-related issues, like us, and wants to expand his perception.

\begin{lstlisting}
process = input : main : output;
\end{lstlisting}

We introduce a three-parted environment: inputs, main, outputs. It is not a
tautological issue, it is a way to insulates the main process by the personalised
environment. Declaring the main group as the only place for the score related
processes, each architectural custom construction useful to staging it must be
outside that region. So, each input channel pre-processing and complex out-mix
are predisposed in the relative paths. By this way, a work-by-work practice is
straightforward looking inside the main boxes, and custom infrastructure remains
the same every time.

\begin{figure}[ht]
\centerline{\includegraphics[width=.45\textwidth]{img/lais-process}}
\caption{\label{re-dia-6c}{\it The three-parted process with input and output
custom infrastructures, independent from the main process. The 18 inputs are
derived from \emph{RME Fireface 800} included in \emph{hardware.lib}.}}
\end{figure}

%-------------------------------------------------------------------------------
%-------------------------------------------------------------------------------
%-------------------------------------------------------------------------------

\section{Rewrite}
\label{sec:rewriting}

The second topology of music score has deep electroacoustic documentation and an
accurate specific musical notation. Nevertheless, the documentation requires a
musicological approach to unveil the meaning of, what we defined, \emph{hole-word}.
\emph{Risonanze Erranti} is a long work of the latest Nono's composition period,
with many live electronics instruments inside the ensemble, some of that was
undocumented hardware, names without meaning, \emph{hole-word}.

%-------------------------------------------------------------------------------

\subsection{1989, \emph{Risonanze Erranti}, Luigi Nono}

%Risonanze erranti, composta nel 1986, ebbe la sua prima esecuzione nel marzo
% dello stesso anno a Colonia, a cui seguirono altre due esecuzioni, Torino 1986
% e Parigi 1987, prima di arrivare alla versione definitiva. Questo lavoro si
% configura come la prima tappa di un ciclo di Lieder che doveva svilupparsi in
% parallelo ai post-prae ludi (il n.1 “per Donau” e il n.3 “BAAB-ARR”),
% composizioni ideate “prima” di Prometeo. Tragedia dell'ascolto (1984-85), ma
% realizzate “dopo” e strettamente legate al virtuosismo dei suoi
% solisti-collaboratori. Il lavoro è dedicato a Massimo Cacciari che ha curato i
% testi di Prometeo e di molti altri lavori di questo periodo, oltre ad aver
% condiviso con Nono lo sviluppo di una nuova fase creativa a cavallo degli anni
% '80 del secolo scorso. In Risonanze erranti, Nono utilizza frammenti di testi
% di Herman Melville, soprattuto dai Battle-Pieces and Aspects of the War (1866)
% e di Ingeborg Bachmann (Kleine Delikatessen, 1963) con echi musicali del
% passato tratti da Guillaume de Machaut (Lay de plour), Josquin Desprez (Adieu
% mes amours) e Johannes Ockeghem (Malheur me bat). Alterna forti contrasti
% dinamici  nelle percussioni con colpi secchi dei bongos e dei crotali che
% diventano carezze sonore quando i percussionisti sfiorano con le mani la
% superficie rugosa delle campane di pastori sardi, la pelle dei tamburi, i
% dischi di metalo dei crotali. Queste sonorità subliminali vengono ulteriormente
% moltiplicate e proiettate nello spazio acustico attraverso l'elettronica, con un
% banco di 8 echi elettronici caratterizzati da una precisa struttura ritmica
% asimmetrica nella sua ripetizione iterata: nelle parole di Nono, “suoni erranti
% nello spazio vero strumento componente sempre più in attesa”. In maniera
% analoga la voce si interpola con il flauto/ottavino e la tuba, confondendosi a
% vicenda, esplodendo in sforzatissimi a cinque f per sparire nel silenzio sonoro
% dei pianissimi a sei p, alternando gesti esasperati a rassegnati abbandoni in
% cui la parola adieu, da Desprez, si allontana nello spazio come fosse lanciata
% verso l'infinito.
%
%Alvise Vidolin
%
%(tratto dal programma di sala dell'Ex Novo Musica 2015, Classici di Oggi, Venezia 2015)

\begin{figure}[ht]
\centerline{\includegraphics[width=.45\textwidth]{img/re-diagramma11}}
\caption{\label{re-dia-6c}{\it Block diagram of scene 11a and 11b}}
\end{figure}

To avoid misunderstanding, every technological rewriting based on block-diagram
must be a partial true. Each block named with an intergalactic hole-word can
lead everywhere. The sound of the \emph{Halaphon} (to cite one of the Nono's
\emph{hole-word} block) not exist. The \emph{Halaphon} was a way to connect pure
musical thinking with consolidated musical practice, embracing acoustical space
and electronics. Before the instrument itself, the \emph{Halaphon} was an idea of
space-related music, that became a necessity, and only at the end an opportunity.

\begin{quote}
The Halaphon is a digital spatializer which, with the loudspeakers arranged in the
room, controls the movement of sound in space. This movement must be continuous,
with a soft, superimposed fading from one loudspeaker to another. The dynamics
indicated in the score for the Contralto and the instruments also apply for the
dynamics of the Halaphon outputs \cite{nlre87}. %Overall four interventions with
%the Halaphon are foreseen (see Diagrams 11, 13, 16 and 19): with the exception
%of the intervention of programme 16, they are all closely linked to the
%rhythmic-temporal execution of the voice or instruments. Further information on
%programming the Halaphon is to be found further on, in the paragraph Special
%Information, in the diagrams and notations in the score.
\end{quote}

Words can define and reduce some issues to integrate diagrams and musical notation
where they are lacunose or obscure. Reading a musical score unknowing the mental
state of the composer that brought it to the world is a daily committed crime. A
composer poetics is inevitably invoked into the work he is producing and in his
musical practice and research. Even when there are knowledge and structured
thinking, even then, we can produce the wrong questions to obtain the right, not
necessary, answer.

\begin{quote}
The reverberation effect is only given a duration (is it a plate or spring reverb?
are there any filter or early reflection settings?). We elected to use a high
quality Lexicon reverb as we believed that this worked best musically. In other
areas detailed information is given, such as precise frequencies/bandwidths for
the two filter banks used \cite{rw05}.
\end{quote}

In other areas, detailed information is given, such as precise frequencies/bandwidths
for the two filter banks used, because they are musical related instrument
information. An example of a not musical related instrument information is in
\emph{Risonanze} score:

\begin{quote}
This reverberation time only occurs in connection with the voice of the Contralto,
and is always replayed on loudspeakers L9, 10, placed in the middle of the room,
at the top. Depending on the acoustics of the hall, it may be increased to 5
seconds: However, this decision should only be taken in relation to the duration
of other reverberations \cite{nlre87}.
\end{quote}

The difference between what could be instrument or not is very clear: the
articulated performability through listening. A frequency/bandwidth is a clear
musical reference, as the pitch. The sound produced by a typology of reverb is
an architectural choice, as the choice of the space in which stage the piece.

\begin{quote}
Io entro nello Studio di Freiburg, sempre, "senza idee". Senza programmi. Questo
è fondamentale perché significa l'abbandono totale del logocentro, la perdita di
quel principio per cui sempre un'idea dovrebbe essere antecedente alla musica.
L'idea come ciò che deve essere realizzato o espresso nella musica. Oppure la
storia che deve essere raccontata "in musica". [\ldots] %Nello Studio - ho detto -
%si entra. Ci sono strumenti musicali a disposizione e si comincia ad agire in
%due ordini di metodo diversi: il primo è quello vero e proprio della fisica
%acustica. [\ldots] Abbiamo, a Freiburg, tanti tipi di computer (uno
%particolarissimo, appena arrivato, ancora non lo conosciamo).
Lavoriamo nello Studio come se fossimo Gnostici: intuizione immediata, mediata,
strumentazione, ricerca. È stata la conoscenza del filosofo olandese Brouwer a
introdurmi [\ldots] la necessità della "percezione della mutazione". Stiamo
vivendo un'epoca di continue mutazioni, trasformazioni, frantumazioni
\cite{nono85}\footnote{I always enter in the Freiburg's Studio “without ideas”.
Without any programs. This is fundamental because it signifies the total abandon
of the logocentrism, the loss of that principle which establishes that any idea
must come always before music. The idea as what has to be realised or expressed
through music. Or the story that has to be narrated “in music”. [\dots] We work
in the Studio as we were gnostics: with immediate intuition, mediate intuition,
instrumentation and research. It has been the knowledge of the Dutch philosopher
Brouwer to introduce me [\dots] the necessity of the “perception of the mutation”.
We are living in an era of continuous mutations, transformations and fragmentations.}.
\end{quote}

If there is something that must be sustained is exactly that musical behaviour.
Each of those Nono's words conducts the musician to agile and deeply performable
electroacoustic musical environments. So it passes the concept of an instrumental
practice consolidated on the means and tools available. Nono himself talks about
it by transversally crossing architecture, classical musical practice and
technology, in executive and interpretative terms:

\begin{quote}
Lo spazio è uno degli elementi con cui componi, anche se dall'Ottocento, dal tempo
della sala da concerto e dell'opera, ciò non succede più.
Tutto il melodramma italiano si è realizzato in una forma già prefissata. Ma
continuare così sarebbe stato come considerare vera la sola forma sonata di un
certo periodo della vita di Beethoven, come se lui non avesse continuamente
trasformato e stravolto quella forma fino alle ultime sonate.

Questo vuol dire, per me, pensare la musica. E la stessa cosa avviene col computer:
nel tempo reale tu hai la possibilità di programmare, ma anche di intervenire,
modificare, trasformare tutto, completamente. Una volta programmato, il computer
non va avanti come una locomotiva sul binario, che niente la può fermare. Il
computer non è intelligenza delegata agli altri. No, è un mezzo che ti obbliga a
un nuovo tipo di sapere, di conoscenza, [\ldots] %esattamente come i piani
%acustici della chiesa di S. Lorenzo. Intendo dire che Piano ha costruito,
%insieme alla chiesa, una \emph{machina da sonàr} come si diceva nel Cinquecento.
%E con lo Studio di Friburgo, con Hans Peter Haller, con Alvise Vidolin, con il
%processore, le quattro orchestre e i solisti,
noi verifichiamo continuamente le acustiche e inseriamo delle continue modifiche
a ciò che ho pensato o scritto \cite{nono84}\footnote{The space is one of the
composing elements, even if from the nineteenth century, from the concert hall
time and from the opera, it does not happen anymore. The italian melodrama has
been realised with a prefixed construction form. To continue as it was thought
initially, it could have been like we have only considered truthful the only
Sonata form a certain specific Beethoven's composition period, as he never
transformed and twisted the form till the last sonatas. This is, for me,
thinking the music. And the same thing happens with the computer: we have the
possibility to program, to intervene, to modify and to transform everything
completely, in real-time. Once a computer is programmed, it will not go forward
as a locomotive on a railway, which it become unstoppable. The computer is not a
delegated intelligence. No, the computer is the means that obliges you to learn
new types of knowledge, [\ldots] we verify continuously the acoustics and we
insert changes in what I have written and thought.}.
\end{quote}

Musical score thinking and annotating procedure to obtain a musical match, like
the Nono's procedure of studying, practising, listening and writing once between
infinite possibilities; declare itself as an attempt to do something not complete
or not fully defined in its process. The division by scenes of the routing, in
the technical score, is a procedure derived by the environment at their disposal.
As we can see through the scenes navigation, they are often a solution to an
old-complexity routing, but today most of these scenes could be joined and driven
by more flexible and accurate, multiple and automate remote controllers.

%\begin{figure}[ht]
%\centerline{\includegraphics[width=.45\textwidth]{img/re-diagramma6c}}
%\centerline{\includegraphics[width=.45\textwidth]{img/re-diagramma3b}}
%\caption{\label{re-dia-6c}{\it Block Diagram 6c and 3b. The 6c structure is
%contained into 3b in a sort of redundancy of subdivision for a today practice.}}
%\end{figure}

%-------------------------------------------------------------------------------
%-------------------------------------------------------------------------------
%-------------------------------------------------------------------------------

\section{Porting}
\label{sec:porting}

The porting of aged music informatics of experimental instruments to a sustained
programming language and technology merging into two branches of interests of the
authors: the history of instruments (even the technological ones) and the revival
of music lost in the past for technological issues, into a new possibility of
music playing.

%--------------------------------------------------------------------------------------------------------------------------------------
\begin{figure}[ht]
\centerline{\includegraphics[width=.45\textwidth]{img/lmfly10}}
\caption{\label{ml-fly10}{\it \emph{FLY10} Module Diagram. The modules are
connected to the host interface at one side, and the DAC at other. Up to two
parallel cards that allow the signal processing, a 12-bit DAC converter, a series
of four cells of the 2nd order Butterworth Low Pass Filter, with two selectable
cut-off frequencies at $4.5KHz$ or $9.3KHz$.
}}
\end{figure}

\subsection{1991, \emph{Mobile Locale}, Michelangelo Lupone}

Working side by side with Michelangelo Lupone for the \emph{Mobile Locale}
\cite{lmml91} porting is something extremely musical related and only marginally
a technological and informatics matter. The main goal is the possibility to
interpret his music, with his unavoidable sensibility at disposal of better
comprehension of the music score. Close to this, the fascinating possibility to
revive a beautiful work, \emph{Mobile Locale}, stuck by technical problems that
obscured its musical value. The work was conceived around a technology born from
the same Lupone's musical thinking, at CRM\footnote{\url{http://www.crm-music.it/}}
(Music Research Centre, in Rome), the \emph{System Fly} \cite{ml85}.

%The Fly 30 system was conceived by Lupone and developed by a CRM Research Team
%about five years after Lupone's Fly 10.

\begin{quote}
The different cultural and professional backgrounds of the members of the team
(musicians, physicists, engineers and musicologists) united by a common
appreciation of interdisciplinary problems and all specialising in informatic
systems, produced the group's present capacity to create a digital system
oriented towards the synthesis, analysis and real-time processing of sound
signals. This system develops an immediate interaction with the user and it is
flexible and adaptable to the different scientific and artistic needs \cite{ml91}.
\end{quote}

The hardware and software development between the two systems, the \emph{Fly 10}
and the \emph{Fly 30}, was oriented on real-time musical interaction as, during
those years, was pioneered by the institutionalized research labs. At the same
CIM\footnote{\url{http://www.aimi-musica.org/?page_id=13}} event in which Lupone
described the System Fly \cite{ml85},  Sylviane Sapir and Alvise Vidolin
\cite{savi85} portray, with inspiring words, the real-time \emph{Prometeo}
production

\begin{quote}
Con l'avvento dei sistemi informatici in tempo reale è diventato possibile
ri-mettere in diretta corrispondenza il gesto con il suono ricreando il feed-back
azione-suono-ascolto-azione e, di conseguenza, far uscire gli elaboratori dai
laboratori e concepire delle composizioni che prevedessero una esecuzione dal
vivo sia autonoma sia integrata con altri strumenti tradizionali e
non\footnote{With the appearance of real-time informatics systems, it has become
possible to re-put the gesture and the sound in direct correspondence, recreating
the action-sound-listening-action feed-back and, consequently, bringing the
computers out of the laboratories and conceiving compositions that provided for a
live performance both autonomous and integrated with other traditional and
non-traditional instruments.}.
\end{quote}

Knowing the structure of \emph{Fly 10}, shown in figure \ref{ml-fly10}, and its
procedure to generate sound at the electric stadium is fundamental to understanding
some of the composer tunings of algorithms,  simultaneously to balance the new
real-time processing made by the porting mixed with the original tape sound made
with the \emph{Fly 10} system. The 1991 state of the art of the \emph{System Fly
30} structure had $16Bit$ ADC/DAC with a sample rate up to $96KHz$. For the
release of Mobile Locale, the label \emph{Edipan} mastered the tape with upsampling
to $16Bit$ at $48KHz$. The sounds generated with today real-time architecture
could be in high resolution and at higher sampling frequencies. Discussing with
Lupone all those historical data and the new procedure, underlined the necessity
to have a set of filters, similar to those at the end of \emph{Fly} systems, to
search a timbral balancing during the staging.

The entire electronics, both tape and live, was conceived as a shadow of the
acoustic percussions and amplification of them. So there are three levels of
musical matter on stage: the acoustic (a complex set of percussions), the
electroacoustic (sound reinforcement, early reflections simulation and tape),
the live electronics (real-time processes based on three different delay lines
usages).

\begin{figure}[ht]
\centerline{\includegraphics[width=.45\textwidth]{img/1-comp}}
\caption{\label{ml-gen-dia}{\it General block diagram in score instructions.}}
\end{figure}

During the staging strategy, the entire sound direction must be focused on the
balancing of the live electronics with the electroacoustic sound, that has to be
hidden, by all of these, inside the percussion sounds\footnote{Personal
conversation with the composer.}.

There are four places, areas, described in the general block diagram of the
algorithm, respectively: the \emph{QAQF} Delay line with the sinusoidal oscillating
reading index, the comb filtering \emph{Early Reflections}, the scaled mix of
the reflection into one channel and the two feedback delay lines, with different
step-index, named \emph{WA} and \emph{ZA}.

The \emph{QAQF} is a delay line read by an oscillator pointer at some fixed rates
that, at the higher value of 320Hz, produces a broadband signal that must be
treated by the output filters discussed before.

\begin{figure}[hb]
\centerline{\includegraphics[width=.45\textwidth]{img/main}}
\caption{\label{ml-main}{\it The frame of main processes. It is easy with Faust
to group the code to obtain a logical diagram similar to the scored one. }}
\end{figure}

\begin{figure}[ht]
\centerline{\includegraphics[width=.45\textwidth]{img/2-comp}}
\caption{\label{ml-dia-exp}{\it Score Block Diagram Explosion}}
\end{figure}

The early reflections are simulated by the different delay times of eight comb
filters, projected to the audience by coupling them two by two through the four
channels. In the score instructions, there are not precise gain values of recursion
inside the filters, instead, they must be taken directly from Fly 30 patch source
code.

The two delays named WA and ZA have the feedback subtracted (not summed) with the
input signal. It is necessary to understand the timbral significance of that
choice, in the same perspective of timbral control of the shadowed electronics.

The relationship between instrument, opera and musical idea as the core of Lupone's
composition process, brings again the conversation at the necessity to have a
musical instrument, not only an electronic environment, mastered to focus on
expected musical behaviour.



\begin{figure}[ht]
\centerline{\includegraphics[width=.45\textwidth]{img/wa}}
\caption{\label{wa-block}{\it WA Block Diagram with the macro on the signal
crossing before subtraction. In the sum, the order of inputs doesn't matter,
\emph{Faust} logically feedback into the first signal, but in the difference is
necessary to properly ordinate the two.}}
\end{figure}

%\begin{figure*}[ht]
%\center
%\includegraphics[width=5in]{TwoColumnSine2}
%\caption{\label{ftt_plot2}{\it A figure spanning two columns, as mentioned in
%Sec. \ref{ssec:figures}.}}
%\end{figure*}

%\subsection{Tables}
%
%As for figures, all tables should be centered on the column (or page, if the
%table spans both columns). Table captions should be in italic, precede each
%table and have the format given in Table~\ref{tab:example}.
%
%\begin{table}[ht]
%  \caption{\itshape Basic trigonometric values.}
%	\centering
%	\begin{tabular}{|c|c|}
%		\hline
%		$\mathrm{angle}\,(\theta, \mathrm{rad})$ & $\sin \theta$ \\\hline
%		$\frac{\pi}{2}$ & $1$ \\
%		$\pi$ & $0$ \\
%		$\frac{3\pi}{2}$ & $-1$ \\
%		$2\pi$ & $0$ \\\hline
%	\end{tabular}
%	%
%	\label{tab:example}
%\end{table}
%
%\begin{table*}[ht]
%  \caption{{\it Basic trigonometric values, spanning two columns.}}
%	\centering
%  \begin{tabular}{|c|c|c|c|c|c|c|}\hline
%    $\mathrm{angle}\, (\theta, \mathrm{rad})$ & $\sin \theta$ &
%    $\cos \theta $ & $(\sin \theta)/2 $ & $(\cos \theta) /2 $ &
%    $(\sin \theta)/3 $ & $(\cos \theta)/3$    \\\hline
%    $\frac{\pi}{2}$ & $1$ & $0$ & $1/2$ & $0$ & $1/3$ & $0$ \\
%    $\pi$ & $0$ & $-1$ & $0$ & $-1/2$ & $0$ & $-1/3$\\
%    $\frac{3\pi}{2}$ & $-1$ & $0$ & $-1/2$ & $0$ & $-1/3$ & $0$ \\
%    $2\pi$ & $0$ & $1$ & $0$ & $1/2$ & $0$ & $1/3$ \\\hline
% \end{tabular}
%	%
%  \label{tab:example2}
%\end{table*}
%
%\subsection{Equations}
%
%Equations should be placed on separate lines and numbered:
%
%\begin{equation}
%	y(n)=b_0x(n)-a_1y(n-1)
%	\label{eq1}
%	\end{equation}
%	where equation (\ref{eq1}) is a one pole filter with frequency response:
%	\begin{equation}
%	H(e^{j \omega T}) = \frac{b_0}{1+a_1e^{-j \omega T}}
%	\label{eq2}
%\end{equation}

%\subsection{Code}
%
%Code can be listed in a block:
%
%\begin{lstlisting}
%  int foo = 0;
%\end{lstlisting}
%\noindent
%or directly in-lined in the body of the text: \lstinline{int foo = 1;}.
%
%
%\subsection{References}
%
%The references will be numbered in order of appearance  \cite{Sal89},
% \cite{Spa72},  \cite{MosWal64} and  \cite{Kay86}. Please avoid listing
%references that do not appear in the text.
%
%\subsubsection{Reference Format}
%
%The reference format is the standard IEEE one. We recommend to use BibTeX to
%create the reference list.

\clearpage

\section{Conclusions}

With this article, the music sustainability concept was spread from live
electronics music to the broader electroacoustic music composition and
interpretation. The original issues \cite{bevi05} about music score documentation
in electronic music is the fundamental core of the concept. Nevertheless, the
focus of this research, and the approach, point at a less technical and more
compositional and practical situation that afflicts not only the documentation
of a score but the musical thinking and practice at all.

The research defines different topologies of electroacoustic music (the
\emph{undocumented}, where composers use only words descriptions to generate
environment and circumstances; the \emph{hole-word} where the score has deep
technical documentation but listing names of undocumented instruments; the
\emph{porting}, where informatics translations between languages, or informatics
technologies, are based on literature and shared knowledge) consolidating
emerging critical circumstances: sustainability is only marginally related to the
documentation and it is only superficially a technical issue. The documentation
is a quality parameter of sustainability but it is the musical practising and
interpreting that will build musical thinking during the years.

The first concept to be clarified in the conclusions is that sustainability must
aim at maintaining the musical idea, the peculiarities of the piece and of what
we could define as the \emph{sustainability of the process}. The compositions
here treated points up to this fundamental aspect: the practice on difficulties
raised studying each musical literature work must become the documentable,
sustainable and improvable musical core of the repertoire.

To improve, share and grow the musical interpretation of repertoire there are
rules to be observed, and we derived them by informatics sustainability itself:
\emph{Open and Be Open}, \emph{Don't Repeat Yourself}, \emph{Think and Act as
Community}.

The sustainability process also points at the fact that a community can truly
build instruments one time only, as a tool, and refine it. Making it accessible
through open-source would lead to the interpretation and implementation of
electroacoustic compositions, preserving electronic thinking for a greater
progression. Researching within contemporary composing means untying the
possibilities of realisation from tools and means available during music
composition.

A community can operate as a Research Group, with a "common appreciation of
interdisciplinary problems" \cite{ml91}, to bring a time-frozen composition back
to a warm and discussed work, out of a solipsistic production-in-a-box typical
of the personal computing era, to focus on musical matters carved out on the
personal knowledge, outside the personal point of view.

It is necessary to focus on the main difference between technical sustainability
and musical sustainability. Technical sustainability concerns the work, it is
linked to the technical world that the work defines. It is its carbon dating, the
reproducible ecosystem, maybe, but it is not the work itself. Musical
sustainability is a matter of thoughts, that makes use of those tools to go out
towards the perceptible. Supporting the thoughts is supporting music, perception
and listening.

\begin{quote}
La musica non è solo composizione. Non è artigianato, non è un mestiere. La
musica è pensiero \cite{nono85}\footnote{Music is not only about composing.
It's not artisanship, neither only a craft. Music is the thought.}.
\end{quote}
%\section{Acknowledgments}
%
%Many thanks to the great number of anonymous reviewers!

%\newpage
\nocite{*}
\bibliographystyle{IEEEbib}
%http://cim.lim.di.unimi.it
\bibliography{LAC-20-SEAM} % requires file lac-20.bib
%
%\section{Appendix: Margin Check}
%
%This section shows the column margins for the text.
%
%Lorem ipsum dolor sit amet, consectetur adipisici elit, sed eiusmod tempor
%incidunt ut labore et dolore magna aliqua. Ut enim ad minim veniam, quis
%nostrud exercitation ullamco laboris nisi ut aliquid ex ea commodi consequat.
%Quis aute iure reprehenderit in voluptate velit esse cillum dolore eu fugiat
%nulla pariatur. Excepteur sint obcaecat cupiditat non proident, sunt in culpa
%qui officia deserunt mollit anim id est laborum.

\end{document}
